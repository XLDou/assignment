\documentclass{article}
\usepackage{amsmath}
\usepackage{physics}
\usepackage{listings}
\usepackage{parskip}
\usepackage{siunitx}

\title{Assignment 1}
\begin{document}

\section{Car and Bicycle Rider}
\subsection{a)}
By the fact that $v_c(t) = \int_{t_0}^t a_c(s) ds + v_0$, we have
\begin{align}
  v_c(t) = \begin{cases}
    v_0 & 0 \leq t \leq t_1 \\
    v_0 - \frac{c}{2}(t - t_1)^2 & t1 \leq t \leq t2\\
    0 & t \geq t_2.
  \end{cases} 
\end{align}
Solving $v_c(t_2) = 0$, we get $t_2 = t_1 + \left(2v_0 / c\right)^{1/2}$.

When $t \leq t_1$, it's obvious $x_c(t) = v_0 t$. When $t_1 \leq t \leq t_2$,
we have the following
\begin{equation}
  x_c(t) = x_c(t_1) + \int_{t_1}^{t}v_c(s) ds = v_0t - \frac{c}{6} (t - t_t)^3.
\end{equation}
After $t_2$ the car stays at $x_c(t_2) = v_0t_2 - \frac{c}{6}(t_2 - t_1)^3=v_0 t_1
+\frac{2v_0}{3}\sqrt{\frac{2v_0}{c}}$.

Then we have
\begin{align}
  x_c(t) = \begin{cases}
    v_0 t & t \leq t_1 \\
    v_0 t - \frac{c}{6}(t - t_1)^3 & t_1 \leq t \leq t_2 \\
    v_0t_1 + \frac{2v_0}{3}\sqrt{\frac{2v_0}{c}} & t \geq t_2\\
  \end{cases}
\end{align}

\subsection{b}
Solvint the following equation, for notational simplicity, let $x_b = x_b(0)$.
\begin{align*}
  x_b + v_b t_2  &= v_0 t_1 + \frac{2v_0}{3}\sqrt{\frac{2v_0}{c}}\\
  x_b + v_b(t_1 + \sqrt{2v_0/c}) &= \cdots\\
  v_b &= \frac{-x_b + v_0t_1 + \frac{2v_0}{3}\sqrt{\frac{2v_0}{c}}}{t_1 + \sqrt{2v_0/c}} \\
                 &= (17 + 12 \times 1 + 16) / (1 + 2) = 45 / 3 = 15
\end{align*}


\section{Elevator Trip}
\subsection{a}
\begin{align}
  v(t) = \begin{cases}
    at & 0 \leq t \leq T\\
    aT & T\leq t \leq 5T\\
    aT - a (t - 5T) = 6aT - at & 5T \leq t \leq 6T. 
  \end{cases}
\end{align}
\subsection{b}
Integrating over $v$ we have the height function $y(t)$ as follows
\begin{align}
  y(t) = \begin{cases}
    \frac{t^2}{2a} & 0 \leq t \leq T\\
    aTt - \frac{aT^2}{2} & T \leq t \leq 5T\\
    -13aT^2 + 6aTt - \frac{a}{2}t^2 & 5T \leq t \leq 6T.
  \end{cases}
\end{align}
By solving $y(6T) = h$, we get
\begin{align}
  a = h / 5T^2.
\end{align}


\section{Rocket Launch}
The rocket reaches its maximum height when the velocity becaues zero. We let the
time $t_{1}$ denotes the maximum height time. We have
\begin{align}
  v_r(t) = 
    At - \frac{Bt^2}{2} \quad t \leq t_1.
\end{align}
Then we have
\[t_1  = 2A / B.\]
We also have the height of rocket,
\begin{equation}
  y_r(t) = \frac{At^2}{2} - \frac{Bt^3}{6} \quad t \leq t_1.
\end{equation}
We ball height has the following
\begin{equation}
  y_b(t) = h - \frac{gt^2}{2} \quad t \leq t_1.
\end{equation}
Solving $y_b(t_1) = y_r(t_1)$ and plug in $t_1$, we have
\begin{align}
  h = \frac{1}{B^2}\left(2g\cdot A^2 + \frac{2}{3}A^3\right).
\end{align}


\section{Throw and Catch}
The height difference of the ball is
\begin{equation}
  \Delta h_b(t) = v_0\cdot \sin(\theta)t - \frac{g}{2}t^2.
\end{equation}
By solving $\Delta h = 0$ for a non-trival solution, we get the catch time
\[t_1 = 2\sin(\theta)v_0 / g.\]
The ball x-xias has the following equation
\[x_b(t) = \cos(\theta)v_0 t.\]
The person x-xias has
\[x_p(t) = d + \frac{B}{6}t^3.\]
Solving $x_b(t_1) = x_p(t_1)$, we get
\[B = \frac{3g^3}{4\sin^3(\theta)v_o^3}\left( \frac{2\sin\theta\cos\theta v_0^2}{g} - d\right).\]

\section{Vertical Collision}
\subsection{a}
In general the height of balls after throw has the following:
\[h_b(\Delta t) = v_0 \Delta t - \frac{g}{2}\Delta t^2.\]
The the two times of $h_b(t) = h$ as the roots have
\begin{equation}\label{eq:sum_traval}
 t_ 1 + t_2 = 2v_0 / g.
\end{equation}
Is obvious that $t_i$ is the travel time of $i$-th ball. We know that
\begin{align}
  t_1 = t_h \\
  t_2  + 4 = t_h 
\end{align}
Combine with \eqref{eq:sum_traval}, we get
\[2t_h + 4\si{s} = 2v_0 / g. \]
\subsection{b}
Solving $v_0 t_h - \frac{g}{2}t_h^2 = h$, we can get $v_0$

\end{document}
